% ----------------------------------------
% Chapter: Stand der Technik
% ----------------------------------------
\chapter{Stand der Technik}
\label{chap:stand_der_technik}

Ein geschmierter Reibungskontakt kann auf vier Elemente des tribologischen Systems nach Czichos \cite{czihos} reduziert werden:
\begin{itemize}
    \item Grundkörper
    \item Gegenkörper
    \item Zischenstoff
    \item Umgebungsmedium
\end{itemize}

% ----------------------------------------
% Fig: Das tribologische System
% ----------------------------------------
\begin{figure}[htb]
    \centering
    \includegraphics[width=5cm]{./images/blank_img.jpg}
    \caption{Das tribologische System}
    \label{fig:das_tribologische_system}
\end{figure}

Um ein besseres Verständnis der EHD-Schmierung zu haben, wird in diesem Abschnitt zuerst die Kennwerte des Zwischenstoffes (Schmiermittel) und der beiden Kontaktelementen (Grund- und Gegenkörper) ausführlich besprochen.
Danach wird der Mechanismus der EHD-Schmierung beleuchtet und am Ende wird die Arbeit von Hamrock und Dowson zur theoretischen Bestimmung der Schmierfilmdicke erwähnt.

% ----------------------------------------
% Sec: Eigenschaften des Schmiermittels
% ----------------------------------------
\section{Eigenschaften des Schmiermittels}
\label{sec:eigenschaften_des_schmiermittels}
Viskosität, die auch als innere Reibung bezeichnet wird, ist die wichtigste Kenngröße eines Schmierstoffes.
Sie beschreibt die Fließfähigkeit und dann reversibele Verformung eines Stoffes unter die Wirkung einer Scherspannung.

\begin{itemize}
    \item Viskosität
    \item Kinematische Viskosität
    \item Temperatureffekt
    \item Einfluss von Druck auf Viskosität
    \item Dichte
    \item Brechungsindex
    \item Wärmeleitfähigkeit
    \item Nichtnewtonsches Verhalten
    \item Verfestigung der Schmierung bei hohem Druck
\end{itemize}

% ----------------------------------------
% Sec: Betrachtung des EHD-Kontaktes
% ----------------------------------------
\section{Betrachtung des EHD-Kontaktes}
\label{sec:betrachtung_des_ehd_kontaktes}

\begin{itemize}
    \item Nichtkonformer Kontakt
    \item Hertzsche Gesetz
        \begin{itemize}
            \item Kugel-Kugel
            \item Kugel-Platte
        \end{itemize}
    \item Kontakt von beschichteten Körpern
\end{itemize}

% ----------------------------------------
% Sec: Elastohydrodynamische Schmiertheorie
% ----------------------------------------
\section{Elastohydrodynamische Schmiertheorie}
\label{elastohydrodynamische_schmiertheorie}

Erklärung, wie EHD funktioniert

% ----------------------------------------
% Sec: Schmierung nach Hamrock und Dowson
% ----------------------------------------
\section{Schmierfilmdicke nach Hamrock und Dowson}
\label{sec:schmierfilmdicke_nach_hamrock_und_dowson}
Erklärung, wie mann die Schmierfilmdicke berechnen kann

