\chapter{Zusammenfassung und Ausblick}
\label{zusammenfassung_und_ausblick}

\begin{itemize}
    \item Die Wichtigkeit von der Schmierfilmdickenmessung?
    \item Ziel dieser Arbeit? (Vergleich optische mit kapazitiver Messmethode. Einfluß der Testparameters auf Messergebnisse)
    \item Was wurde gemacht, um diesen Versuch möglich auszuführen? (konstruktive Bearbeitung)
    \item Versuchsdurchführung
    \item Wie waren die Messergebnisse?
    \item Was kann man bei dem System verbessern? Ist es möglich das System mit Öl zu verwenden?
    \item Der Ausblick. Was können die Ergebnisse von dieser Arbeit sagen?
\end{itemize}

Die Qualität des Schmierfilms spielt eine große Rolle in der Lebensdauer der Komponenten in einer Maschine.
Je größer dieser ist, desto langlebiger ist Maschine.
Zur Untersuchung des Einflußes der Temperatur, der Geschwindigkeit, der Belastung etc. auf den Aufbau des Schmierfilms in einem EHD-Kontakt wurden verschiedenen Versuche an einem Modellprüfstand durchgeführt.
Dabei wird zwischen einer rotierenden Glasscheibe und einer Stahlkugel ein Kontakt im Bereich der elastohydrodynamischen Schmierung erzeugt.
Der Modellprüfstand bildet reale Maschinenelemente nur eingeschränkt ab, bietet allerdings gute Möglichkeiten zur Untersuchung des Verhaltens des Schmierstoffes unter Unterschiedlichen Betriebsbedingungen.

Im Rahmen dieser Arbeit wurde ein neues System entwickelt, das die Messungen mit einer optischen und elektrischen Methode gleichzeitig durchführen kann.
Dafür mussten die Prüflinge (Kugel, Glasscheibe) modifiziert werden.
Da die \textit{Spacer-Layer-Glasscheibe} sich den Strom nicht leitet, wurde sie auf einer Hälfte für den elektrischen Versuch mit einer stromleitenden Chromschicht beschichtet.
Das Kugelsupport wurde neu konstruiert, um die Übertragung der elektronischen Signalen von der Kugel zum Messgerät zu realisieren.
Durch parallele Versuche konnte die Funktion des neuen Messsystems, das die beide Aufbauten --- optisch und elektrisch --- in einem vereinigt, überprüfet werden.

\improvement[inline]{Die Korrelation von beiden Messmethoden unter verschiedenen Testparameter?}

Für weiterführende Arbeiten ist es sinnvoll, die die Messungen bei Fettschmierung mit Variationen von Temperatur, Druck oder Geschwindigkeit durchführen.
Zu diesem Zweck ist es nötig, eine Kammer zur Isolierung des Prüfraums mit der Umgebung zu konstruieren.

