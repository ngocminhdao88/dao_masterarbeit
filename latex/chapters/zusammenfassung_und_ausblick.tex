\chapter{Zusammenfassung und Ausblick}
\label{zusammenfassung_und_ausblick}

\begin{itemize}
    \item Die Wichtigkeit von der Schmierfilmdickenmessung?
    \item Ziel dieser Arbeit? (Vergleich optische mit kapazitiver Messmethode. Einfluß der Testparameters auf Messergebnisse)
    \item Was wurde gemacht, um dieser Versuch möglich auszuführen? (konstruktive Bearbeitung)
    \item Versuchdurchführung
    \item Wie war die Messergebnisse?
    \item Was kann man bei dem System verbessern? Ist es möglich das System mit Öl zu verwenden?
    \item Der Ausblick. Was können die Ergebnisse von dieser Arbeit sagen?
\end{itemize}

Die Qualität des Schmierfilms spielt eine große Rolle von Lebensdauer der Komponenten in einer Maschine.
Je größer er ist, desto langlebiger die Maschine ist.  
Zum Untersuchen des Einflußes von der Temperatur, der Geschwindigkeit, der Belastung, etc. auf den Aufbau des Schmierfilms in einem EHD-Kontakt wurden verschiedenen Versuche an einem Modellprüfstand durchgeführt.
Dabei wird zwischen einer rotierenden Glasscheibe und einer Stahlkugel einen Kontakt im Bereich der elastohydrodynamischen Schmierung erzeugt.
Modellprüfstande bilden reale Maschinenelemente nur eingeschränkt ab, bietet allerdings gute Möglichkeiten zur Untersuchung des Verhaltens des Schmierstoffes unter Unterschiedlichen Betriebsbedingungen.

Im Rahmen dieser Arbeit wurde ein neues System entwickelt, das die Messungen mit optischer und elektrischer Methoden gleichzeitig durchführen kann.
Dafür mussten die Prüflinge (Kugel, Glasscheibe) modifiziert werden.
Da die \textit{Spacer-Layer-Glasscheibe} lässt sich den Strom nicht leiten, wurde sie eine Hälfte mit der stromleitenden Chromschicht für den elektrischen Versuch beschichtet.
Das Kugelsupport wurde neu konstruiert, um die Übertragung der elektronischen Signalen von der Kugel zum Messgerät zu realisieren.
Durch parallelen Versuche konnte die Funktion des neuen Messsystems, das die beide Aufbauten (optisch + elektrisch) in einem vereinigt, überprüfen.

\improvement[inline]{Die Korrelation von beiden Messmethoden unter verschiedenen Testparameter?}

Es ist sinnvoll für weiterführende Arbeiten, die die Messungen bei Fettschmierung mit Variationen von der Temperatur, dem Druck oder der Geschwindigkeit durchführen.
Zu diesem Zweck ist es nötig, eine Kammer zur Isolierung des Prüfraums mit der Umgebung zu konstruieren.

