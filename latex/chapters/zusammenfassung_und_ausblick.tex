\chapter{Zusammenfassung und Ausblick}
\label{zusammenfassung_und_ausblick}

Die Qualität des Schmierfilms spielt eine große Rolle für die Lebensdauer der Komponenten in einer Maschine.
Je größer diese ist, desto langlebiger ist die Maschine.
Zur Untersuchung des Einflußes der Temperatur, der Geschwindigkeit, der Belastung etc. auf den Aufbau des Schmierfilms in einem EHD-Kontakt wurden verschiedenen Versuche an einem Modellprüfstand durchgeführt.
Dabei wird zwischen einer rotierenden Glasscheibe und einer Stahlkugel ein Kontakt im Bereich der elastohydrodynamischen Schmierung erzeugt.
Der Modellprüfstand bildet reale Maschinenelemente nur eingeschränkt ab, bieten allerdings gute Möglichkeiten zur Untersuchung des Verhaltens des Schmierstoffes unter unterschiedliche Betriebsbedingungen.

Im Rahmen dieser Arbeit wurde ein neues System entwickelt, das die Messungen mit einer optischen und elektrischen Methode gleichzeitig durchführen kann.
Dafür mussten die Prüflinge (Kugel, Glasscheibe) modifiziert werden.
Da die \textit{Spacer-Layer-Glasscheibe} den Strom nicht leitet, wurde sie auf einer Hälfte für den elektrischen Versuch mit einer stromleitenden Chromschicht beschichtet.
Das Kugelsupport wurde neu konstruiert, um die Übertragung der elektronischen Signalen von der Kugel zum Messgerät zu realisieren.
Durch parallele Versuche konnte die Funktion des neuen Messsystems, das die beide Aufbauten --- optisch und elektrisch --- in einem vereinigt, überprüft werden.

Die Messungen zur Überprüfung der Funktion des neuen Systems wurden bei drei unterschiedlichen Temperaturen --- \SI{40}{\degreeCelsius}, \SI{60}{\degreeCelsius} und \SI{80}{\degreeCelsius} --- durchgeführt.
Die Vergleichsmessung hat bewiesen, dass die optische Funktion des neuen Systems gute Ergebnisse sowie die von der Firma \textit{PCS Instruments} liefern kann.
Bei den elektrischen Messungen funktioniert das neue relativ gut.
Die Messwerte zeigen den Trend ähnlich sowie analytische Lösung, dass die gemessenen Kapazitäten bei steigender Temperatur abnehmen.
Allerdings stoßt das System bei hohen Geschwindigkeiten aufgrund einer niedrigen Auflösung seine Grenze auf.

Für weiterführende Arbeiten ist es sinnvoll, die Kabel und elektronischen Anschlüsse und Messgeräten gegen äußerlichen Störungen abgeschirmt zu werden.
Eine dünnere Beschichtung würde vielleicht die Lebensdauer der Chromschicht verlängern.
Es ist auch interessant, die Messungen bei Fettschmierung mit Variationen von Temperatur, Druck oder Geschwindigkeit durchzuführen.
Zu diesem Zweck ist es nötig, eine Kammer zur Isolierung des Prüfraums mit der Umgebung zu konstruieren.

