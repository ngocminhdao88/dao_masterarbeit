\chapter{Einleitung}
\label{chap:einleitung}

Eine Maschine besteht im Allgemein aus beweglichen Teilen.
Dabei tritt es zwischen den sich berührenden Oberflächen Reibung und Verschleiß auf.
Um diese unerwünschte Effekte gering zu halten, werden diese Teile in Maschine gelagert.
Es gibt verschiedene Type vom Lager, aber ein weit verbreitetes Maschinenelement ist das Wälzlager.
Das Geheimnis bei einem sicheren und langlebigen Betrieb eines Wälzlagers ist die Schmierung.
Die kann in Form einer Fettschmierung oder einer Ölschmierung sein.

Im Wälzlager entsteht zwischen den Oberflächen von relativen zu einander beweglichen Teilen einen Schmierfilm.
Unter der enormen Belastung im kontraformen Kontakt befindet sich dieser Schmierfilm im Bereich der elastohydrodynamischen Schmierung.
Für die Ölschmierung steht viele Möglichkeiten zur Bestimmung der Schmierfilmdicke, wie zum Beispiel analytische, optische, kapazitive, taktile Methode, etc.
Jede Methode hat eigene Vorteile und Nachteile, aber wurden sie aufgrund technischen Aufwand und der Eigenschaften der Test-Maschinenelementen meist separat ausgeführt.
Ist es möglich, die Vorteile von verschiedenen Messmethoden, nämlich das optische und elektrische Messverfahren zu kombinieren?

Um diese Frage zu beantworten, wird ein System im Rahmen dieser Arbeit entwickelt, das beide Aufbauten vereinigt.
Die Entwicklung basiert auf einem Kugel-Scheibe-Modellprüfstand.
In dem wird zwischen einer rotierenden Glasscheibe und einer Stahlkugel ein Wälzkontakt, der etwa ähnlich mit dem im realen Wälzlager ist, hergestellt.
Um die optische und elektrische Schmierfilmdickenmessung gleichzeitig in einem System auszuführen, wird der Prüfstand teilweise modifiziert.

