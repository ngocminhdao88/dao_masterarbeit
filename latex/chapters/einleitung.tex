\chapter{Einleitung}
\label{chap:einleitung}

Eine Maschine besteht im Allgemein aus beweglichen Teilen.
Dabei tritt zwischen den sich berührenden Oberflächen Reibung und Verschleiß auf.
Um diese unerwünschten Effekte gering zu halten, werden diese Teile in der Maschine gelagert.
Es gibt verschiedene Typen von Lagern.
Ein weit verbreitetes Maschinenelement ist das Wälzlager.
Das Geheimnis für einen sicheren und langlebigen Betrieb eines Wälzlagers ist die Schmierung.
Diese kann in Form einer Fettschmierung oder einer Ölschmierung vorliegen.

Im Wälzlager entsteht zwischen den Oberflächen von relativ zu einander beweglichen Teilen ein Schmierfilm.
Unter der enormen Belastung im kontraformen Kontakt befindet sich dieser Schmierfilm im Bereich der elastohydrodynamischen Schmierung.
Für die Ölschmierung gibt es viele Möglichkeiten zur Bestimmung der Schmierfilmdicke, wie zum Beispiel die analytische, optische, kapazitive, taktile Methode etc.
Jede Methode hat eigene Vorteile und Nachteile, jedoch wurden die einzelnen Verfahren aufgrund von technischem Aufwand und der Eigenschaften der Test-Maschinenelementen meist separat ausgeführt.
Ist es möglich, die Vorteile von verschiedenen Messmethoden, nämlich die hohe Auflösung des optischen und die einfache Einsetzbarkeit des elektrischen Messverfahren zu kombinieren?

Um diese Frage zu beantworten, wird im Rahmen dieser Arbeit ein System entwickelt, das beide Aufbauten vereinigt.
Die Entwicklung basiert auf einem Kugel-Scheibe-Modellprüfstand.
In diesem wird zwischen einer rotierenden Glasscheibe und einer Stahlkugel ein Wälzkontakt, der etwa ähnlich mit dem im realen Wälzlager ist, hergestellt.
Um die optische und elektrische Schmierfilmdickenmessung gleichzeitig in einem System auszuführen, wird der Prüfstand teilweise modifiziert.

