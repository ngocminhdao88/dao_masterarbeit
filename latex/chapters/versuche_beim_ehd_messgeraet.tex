% ----------------------------------------
% Chap: Versuche auf dem EHD-Messgerät
% ----------------------------------------
\chapter{Versuche auf dem EHD-Messgerät}
\label{chap:versuche_auf_dem_ehd_messgeraet}

% ----------------------------------------
% Sec: Versuchte Öle
% ----------------------------------------
\section{Versuchte Öle}
\label{sec:versuchte_oele}

Um die Funktionsfähigkeit des neuen Messsystem zu überprüfen, wird es mit verschiedenen Schmierstoffe getestet.

\paragraph{Schmierstoff 1}
\label{par:schmierstoff_1}

\begin{itemize}
    \item Firma
    \item Grund für das Öl
    \item Anwendungsbereiche
    \item technische Daten, kinematische Viskosität bei 40 und 100 Grad Celsius
\end{itemize}

\paragraph{Schmierstoff 2}
\label{par:schmierstoff_2}

\begin{itemize}
    \item Firma
    \item Grund für das Öl
    \item Anwendungsbereiche
    \item technische Daten, kinematische Viskosität bei 40 und 100 Grad Celsius
\end{itemize}

% ----------------------------------------
% Sec: Kapazitive Messgeräte zur Schmierfilmdickenbestimmung
% ----------------------------------------
\section{Kapazitive Messgeräte zur Schmierfilmdickenbestimmung}
\label{sec:kapazitive_messgeraete_zur_schmierfilmdickenbestimmung}

% ----------------------------------------
% Sub: Stromladekurve Messgerät
% ----------------------------------------
\subsection{Stromladekurve Messgerät}
\label{sub:stromladekurve_messgeraet}

Im IMKT gibt es ein mobiles Messsystem zur Schmierfilmdickenmessung mittels kapazitiven Messverfahren.
Das System besteht aus einem Laptop, der mit \textit{Laderkurve-Software} installiert, und einer Messkarte (Typ USB-6211) von der Firma National Instrument.

Bei kapazitiver Messmethode wird der EHD-Kontakt als ein Kondensator betrachtet.
Die Auflade des Kondensators erfolgt durch eine Ladespannung $U_L$ ($0,2 - 0,5 \ V$) und einen Vorwiderstand $R_L$ ($1015 \ k\Omega$).
Eine unendliche Erhöhung der Ladespannung hat leider eine negative Wirkung für die Messergebnisse, weil es bei hoher Spannung durch die Ionisierung des Schmierstoffes Rauschen verursacht.
Abbildung \ref{fig:Schematischer_aufbau_des_mobilen_messsystems} zeigt den schematischen Aufbau des mobilen Messsystems zur Schmierfilmdickenmessung bei IMKT an.
% ----------------------------------------
% Fig: Schematischer Aufbau des mobilen Messsystems
% ----------------------------------------
\begin{figure}[htb]
    \centering
    \includegraphics[width=4cm]{./images/blank_img.jpg}
    \caption{Schematischer Aufbau des mobilen Messsystems bei IMKT}
    \label{fig:Schematischer_aufbau_des_mobilen_messsystems}
\end{figure}

Das ganze Messsystem wird von der \textit{Laderkurve-Software} gesteuert.
Sie miss nicht direkt die Kapazität, sondern nimmt sie die Antwort bzw. Ladekurve des ``Kondensators'' auf, danach wird die Auswertung mit einem Matlab-Skript ausgeführt.
Die Software kann die Ladekurve in zwei Modi: \emph{Anzahl der Messwerte} oder \emph{Messung nach Zeit} aufnehmen.
Im Rahmen dieser Arbeit werden alle Messungen mit dem ersten Modus gemacht.
Abbildung \ref{fig:gui_der_laderkurve_software} zeigt die Benutzeroberfläche der Ladekurve-Software bei einer Testmessung mit einem Referenz-Kondensator ($3,3 \ nF$) an.
% ----------------------------------------
% Fig: GUI der Ladekurve-Software
% ----------------------------------------
\begin{figure}[htb]
    \centering
    \includegraphics[width=4cm]{./images/blank_img.jpg}
    \caption{Benutzeroberfläche der Laderkurve-Software}
    \label{fig:gui_der_laderkurve_software}
\end{figure}

Die Software ist relative einfach zu bedienen, allerdings gibt es folgende Punkte, auf die man beachten muss.
\begin{description}
    \item[Anzahl der Messwerte] ist die Auflösung der Messung.
        Zu niedriger Wert kann zum Messfehler führen und zu hohen Wert kann es schnell den Speicher voll machen.
        Der Standardwert ist $2500$.

    \item[Anzahl der Ladekurven] ist die Anzahl der Messungen, die nach einander durchgeführt werden. ist die Anzahl der Messungen, die nach einander durchgeführt werden.
        Der Standardwert ist $10$.

    \item[Abtastrate] ist die Anzahl der Messwerte, die Messkarte pro Sekunde messen kann.
        Die USB-6211 Messkarte von NI kann maximal $250 \ kS/s$, das entspricht $2500$ Messwerte in einem Zeitraum von $10 \ ms$.

    \item[Ladespannung] ist die Spannung zwischen zwei Terminal des Kondensators.
        Der Wert sollte im Bereich von $0,2$ bis $0,5 \ V$ liegen.

    \item[Verzögerung] ist die Wartezeit, die die Software warten muss, bevor sie eine Messung ausführt.
        Der Standardwert ist $1 \ ms$.

    \item[Entladezeit] ist die Zeit zwischen zwei Messungen.
        Sie ist notwendig, um der Kondensator komplette leer bevor jeder neuen Messung zu entladen.
        Der Standardwert ist $100 \ ms$.

    \item[Ladewiderstand] ist der Wert des Vorwiderstands.
        Er ist nur für den Dokumentationszweck und wird in der Messdatei geschrieben.

    \item[Störkapazität] ist die externe Störung, wie zum Beispiel von Messkabel oder statische Kapazität zwischen Messkörpern.
        Dieser Wert wird für die spätere Auswertung verwendet.

    \item[Austecken des Netzteils] ist notwendig, um die Störungen von anderen elektronischen Geräte zu vermeiden.
\end{description}

%
% ----------------------------------------
% Sub: RLC Messgerät
% ----------------------------------------
\subsection{RLC Messgerät}
\label{sub:rlc_messgeraet}

% ----------------------------------------
% Sec: Versuchdurchführung
% ----------------------------------------
\section{Versuchdurchführung}
\label{sec:versuchdurchfuehrung}
