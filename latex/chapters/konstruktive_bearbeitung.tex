% ----------------------------------------
% Chap: Konstruktive Bearbeitung
% ----------------------------------------
\chapter{Konstruktive Bearbeitung}
\label{chap:konstruktive_bearbeitung}

Um die Messgenauigkeit der optischen Interferometrie-Technik mit der einfachen Adaptierbarkeit an verschiedenen realen Maschinenelementen der elektrischen Messmethode zur Schmierfilmdickenmessung im EHD-Kontakt zu vereinigen, soll ein neues modular Messsystem auf Basis des EHD-Prüfstands von PCS entwickelt werden.
Dies System soll die optische und elektrische (kapazitive) Messung gleichzeitig erlauben.
Für solches System gibt es die folgende Anforderungen, die durch konstruktive Bearbeitung in nächsten Abschnitten gelöst werden.
\begin{itemize}
    \item Elektrische Isolierung der Glasscheibe und der Kugel mit dem gesamten System.
    \item Elektrische Zugänge für die Messproben an der Scheibe und der Kugel.
    \item Beschichtung auf der Scheibe, die elektrische und optische Messung gleichzeitig erlaubt.
\end{itemize}

% ----------------------------------------
% Sec: Konstruktion der Kugelführung
% ----------------------------------------
\section{Konstruktion der Kugelführung}

Bei der standardmäßigen Kugelführung von PCS wird eine durchgebohrte Kugel in einem Adapter eingeklemmt, welche dann über einen Querstift mit der Motorausgangswelle des zweiten Antriebs formschlüssig verbunden wird.
Für die kapazitive Messung soll die Kugel elektrisch mit dem gesamten System isoliert werden, das erfolgt durch die Verwendung einer Kunststoffwelle.
Die Kugelaufnahme wird aus Messing gefertigt, um den Kontaktwiderstand mit den zu Signal übertragenden Kohlebürsten zu reduzieren (Abbildung \ref{fig:aufbau_der_neuen_kugelfuehrung}).
% ----------------------------------------
% Fig: Aufbau der neuen Kugelführung
% ----------------------------------------
\begin{figure}[htb]
    \centering
    \includegraphics[width=4cm]{./images/blank_img.jpg}
    \caption{Aufbau der neuen Kugelführung}
    \label{fig:aufbau_der_neuen_kugelfuehrung}
\end{figure}
%

Die geführte Kugelachse ermöglicht die Versuche mit Schlupf und verhindert auch das unkontrollierte Einbringen des Schmierstoffes (Öl, Fett) in den Kontakt.
Der Kraftfluss zwischen dem zweiten Motor und der Kugel kann durch den Wegfall des Querstifts unterbrochen werden.
In diesem Fall dreht sich die Kugelführung frei in der Motorausgangswelle.

% ----------------------------------------
% Sec: Konstruktion des Kugelsupports
% ----------------------------------------
\section{Konstruktion des Kugelsupports}

% ----------------------------------------
% Sec: Die Glasscheibebaugruppe
% ----------------------------------------
\section{Die Glasscheibebaugruppe}
