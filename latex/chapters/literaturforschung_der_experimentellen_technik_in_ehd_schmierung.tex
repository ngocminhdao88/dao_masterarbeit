\chapter{Literaturforschung der experimentellen Techniken zur Schmierfilmdickenmessung in EHD-Kontakten}
\label{chap:literaturforschung_der_experimentellen_technik_in_ehd_schmierung}
% ----------------------------------------
% Optische Methode
% ----------------------------------------
\section{Optische Messung der EHD Schmierfilmdicke}
\label{sec:optische_messung_der_ehd_schmierfilmdicke}

\subsection{Licht Interferometrie}
\label{ssec:licht_interferometrie}

\subsection{Variante von der klassichen optischen Interferometrie Methode}
\label{ssec:variante_interferometrie}

% ----------------------------------------
% Elektrische Methode
% ----------------------------------------
\section{Elektrische Messung der EHD Schmierfilmdicke}
\label{sec:elektrische_messung_der_ehd_schmierfilmdicke}

Neben der optischen Messmethoden gibt es noch die elektrische Messmethode zur Untersuchung der EHD-Schmierung, wie zum Beispiel Widerstand, Kapazität, Entladespannung.
Der Vorteil dieser Methode ist, dass die sie direkt bei Maschinenelementen, welche aus Stahl sind, während Betrieb verwendet werden kann.
Allerdings gibt es auch Nachteile.
Die Form des Kontakts, wo die große lokale Verformung stattfindet, ist nur vermutet und stark vereinfacht.
Die hat den Einfluss auf die Widerstand-,Kapazität-Messergebnisse.
Ein Faktor noch ist die Sauberkeit des Schmierstoffes, welche schwer zu kontrollieren ist.
Generell liefern die elektrische Methode nur die mittlere Werte über den Kontaktbereich und gibt leider keine direkte Indikation der Form des Schmierfilms.

Normalerweise werden die elektrische Methoden für folgende Anwendungen verwendet:
\begin{itemize}
    \item Schmierfilmdickenmessung bei bekannten Schmierstoffen
    \item Detektion des Voll-Schmierfilaufbaus im Kontakt der rauhen Öberflächen
    \item Evaluierung des geschmierten Kontakts unter Einfluss des elektrischen Felds
\end{itemize}

\subsection{Kapazitive Methoden}
\label{ssec:kapazitive_methoden}

\subsection{Resistive Methoden}
\label{ssec:resistive_methoden}

% ----------------------------------------
% Alternative Methoden
% ----------------------------------------
\section{Alternative EHD Schmierfilmdicke Messmethoden}
\label{sec:alternative_messmethoden}

\subsection{Ultraschall}
\label{ssec:ultraschall}

\subsection{Laserinduzierte Fluoreszenz}
\label{ssec:laserinduzierte_fluoreszenz}
