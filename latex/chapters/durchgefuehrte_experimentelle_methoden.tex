% ----------------------------------------
% Chap: 
% ----------------------------------------
\chapter{Durchgeführte experimentellen Methoden zur Schmierfilmdickenmessung}
\label{chap:durchgefuehrte_experimentellen_methoden}

% ----------------------------------------
% Sec: 
% ----------------------------------------
\section{PCS Instrument Prüfstand}
\label{sec:pcs_pruefstand}

Zur Schmierfilmdickenmessung wurde ein ``EHL Ultra Thin Film Measurement System'' der Firma PCS-Instrument genutzt (Abbildung \ref{fig:ehl_messgeraet}).
Basiert auf dem Kugel-Scheibe-Modell und optischer Interferenz wird die Schmierfilmdicke im EHD-Kontakt ermittelt.
Durch einen leicht modifizierten Aufbau, ermöglicht das Gerät auch die Bestimmung von Reibkoeffizienten der eingesetzten Schmierstoffe.
Im Folgenden wird kurz auf die einzelnen Komponenten des EHL-Gerätes und die Modifikation im Rahmen dieser Arbeit eingegangen.

% ----------------------------------------
% Fig: Das tribologische System
% ----------------------------------------
\begin{figure}[htb]
    \centering
    \includegraphics[width=4cm]{./images/blank_img.jpg}
    \caption{EHL-Messgerät}
    \label{fig:ehl_messgeraet}
\end{figure}

% ----------------------------------------
% Sec: 
% ----------------------------------------
\subsection{PC und Elektronikeinheit}
\label{sub:pc_elektronikeinheit}

% ----------------------------------------
% Sec: 
% ----------------------------------------
\subsection{Mechanischer Aufbau}
\label{sub:mechanischer_aufbau}

% ----------------------------------------
% Sec: 
% ----------------------------------------
\subsection{Messsystem zur Schmierfilmdickemessung}
\label{sub:messsystem_zur_schmierfilmdickemessung}

% ----------------------------------------
% Sec: 
% ----------------------------------------
\section{Versuchte Öle}
\label{sec:versuchte_oele}

% ----------------------------------------
% Sec: 
% ----------------------------------------
\section{Versuchdurchführung}
\label{sec:versuchdurchfuehrung}
